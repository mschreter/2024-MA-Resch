%%%%%%%%%%%%%%%%%%%%%%%%%%%%%%%%%%%%%%%%%%%%%%%%%%%%%%%%%%%%%%%%%%%%%%%%%%%%%%%%
% template for student thesis, Oct 2019
% 1.version: Keijo Nissen
% 2.version: Hamman de Vaal
% 3.version: Caroline Danowski & Keijo Nissen
% 4.version: Barbara Wirthl
%
%
% build document e.g. in console with
%    latex <file_name> && dvipdfm <file_name>
%    
% if warning "undefined references", add to bib-file and run
%    bibtex <file_name>
%
%%%%%%%%%%%%%%%%%%%%%%%%%%%%%%%%%%%%%%%%%%%%%%%%%%%%%%%%%%%%%%%%%%%%%%%%%%%%%%%%

% template for DIN A4
\documentclass[12pt,a4paper]{article}

%%%%%%%%%%%%%%%%%%%%%%%%%%%%%%%%%%%%%%%%%%%%%%%%%%%%%%%%%%%%%%%%%%%%%%%%%%%%%%%%
%%%% DETAILS ON THESIS - TO BE FILLED !!! %%%%%%%%%
%%%%%%%%%%%%%%%%%%%%%%%%%%%%%%%%%%%%%%%%%%%%%%%%%%%%%%%%%%%%%%%%%%%%%%%%%%%%%%%%

\newcommand{\thesistype}{TypeOfThesis}
\newcommand{\thesistitle}{\LaTeX-template for LNM theses}
\newcommand{\shorttitle}{\LaTeX-template for LNM theses}
\newcommand{\Author}{AuthorName}
\newcommand{\Supervisor}{SupervisorName}
\newcommand{\DateOfSubmission}{01.01.2001}

%%%%%%%%%%%%%%%%%%%%%%%%%%%%%%%%%%%%%%%%%%%%%%%%%%%%%%%%%%%%%%%%%%%%%%%%%%%%%%%%
%%% LEAD UP %%%%%%%%%
%%%%%%%%%%%%%%%%%%%%%%%%%%%%%%%%%%%%%%%%%%%%%%%%%%%%%%%%%%%%%%%%%%%%%%%%%%%%%%%%

% packages that should be included - extend at will
\usepackage{fancyhdr}
\usepackage{mdwlist}
\usepackage{dsfont}
\usepackage{graphicx}
\usepackage{amssymb}
\usepackage{amsmath}
\usepackage{amsthm}
\usepackage{ifthen}
%\usepackage{epstopdf}
%\epstopdfsetup{update} % only regenerate pdf files when eps file is newer
% \usepackage{ngerman} % if in German; provides German hyphenation and direct use of umlauts

% suggested math abbreviations - extend at will
\renewcommand{\vec}[1]{\boldsymbol{#1}}         % for vectors
\newcommand{\mat}[1]{\boldsymbol{#1}}           % for matrices

\newcommand{\dd}{\mathrm{d}}                    % differential d
\newcommand{\pd}{\partial}                      % partial differentiation d

\newcommand{\tsum}{{\textstyle\sum\limits}}     % for small sums in large equations
\newcommand{\pfrac}[2]{\frac{\pd #1}{\pd #2}}   % \pfrac{f(x,y)}{x} for partial derivative of f(x,y) with respect to x
\renewcommand{\dfrac}[2]{\frac{\dd #1}{\dd #2}} % \dfrac{f(x,y)}{x} for total derivative of f(x,y) with respect to x
\newcommand{\grad}{\nabla}                      % for the gradient
\newcommand{\norm}[1]{\| #1 \|}                 % \norm{x} for the norm of x
\newcommand{\abs}[1]{| #1 |}                    % \abs{x} for the absolute value of x

% suggested text abbreviations - extend at will
\newcommand{\comment}[1]{ }  % put aroung regions that are not to be compiled

%
\newcommand{\insertblankpage}{\mbox{}\thispagestyle{empty}\addtocounter{page}{-1}\newpage}
\newcommand{\BibTeX}{$\mathrm{B{\scriptstyle{IB}} \! T\!_{\displaystyle E} \! X}$}

% source of all figures
\graphicspath{{./}}
    % This is the folder where your figures should be. 
    % You can define more folders by inserting  {./additionalPath/}.

% Leere Fu''llseiten
\makeatletter
\renewcommand{\cleardoublepage}{\clearpage\if@twoside \ifodd\c@page\else
  \hbox{}
  \vspace*{\fill}
  \thispagestyle{empty}
  \newpage
  \if@twocolumn\hbox{}\newpage\fi\fi\fi}
\makeatother

%%%%%%%%%%%%%%%%%%%%%%%%%%%%%%%%%%%%%%%%%%%%%%%%%%%%%%%%%%%%%%%%%%%%%%%%%%%%%%%
%%% page layout %%%%%%%%%%%%%%%%%%%%%%%
%%%%%%%%%%%%%%%%%%%%%%%%%%%%%%%%%%%%%%%%%%%%%%%%%%%%%%%%%%%%%%%%%%%%%%%%%%%%%%%
% for details see, e.g. ``http://en.wikibooks.org/wiki/LaTeX/Page_Layout''
\setlength{\topmargin}{0mm}                % space above header (excluding \voffset)
\setlength{\headheight}{12pt}              % height of header
\setlength{\headsep}{8mm}                  % distance between header and text
\setlength{\textheight}{230mm}             % 
\setlength{\footskip}{10mm}                % distance between text and footer, including footer itself

\setlength{\evensidemargin}{0mm}           % distance to inner edge of left page (if twosided)
\setlength{\oddsidemargin}{8mm}            % distance to inner edge of right page (or both if onesided)
\setlength{\textwidth}{150mm}              % 
\setlength{\marginparsep}{0mm}             % 

\setlength{\voffset}{0mm}                  % 
\setlength{\hoffset}{0mm}                  % 

% suggested headers and footers
\pagestyle{fancyplain}
\fancyhf{}
\fancyfoot[C]{\fancyplain{}{\bfseries\thepage}}
\fancyhead[RO]{\fancyplain{}{\Author}}
\fancyhead[LO]{\fancyplain{}{\thesistype:\ \shorttitle}}

\newenvironment{changemargin}[2]{%
\begin{list}{}{%
\setlength{\topsep}{0pt}%
\setlength{\leftmargin}{#1}%
\setlength{\rightmargin}{#2}%
\setlength{\listparindent}{\parindent}%
\setlength{\itemindent}{\parindent}%
\setlength{\parsep}{\parskip}%
}%
\item[]}{\end{list}}

\usepackage{setspace}
\usepackage{helvet}
%\renewcommand{\familydefault}{\sfdefault}

%%%%%%%%%%%%%%%%%%%%%%%%%%%%%%%%%%%%%%%%%%%%%%%%%%%%%%%%%%%%%%%%%%%%%%%%%%%%%%%%
%%%%%%%%%%%%%%%%%%%%%%%%%%%%%%%%%%%%%%%%%%%%%%%%%%%%%%%%%%%%%%%%%%%%%%%%%%%%%%%%
%%%%  	BEGIN DOCUMENT
%%%%%%%%%%%%%%%%%%%%%%%%%%%%%%%%%%%%%%%%%%%%%%%%%%%%%%%%%%%%%%%%%%%%%%%%%%%%%%%%
%%%%%%%%%%%%%%%%%%%%%%%%%%%%%%%%%%%%%%%%%%%%%%%%%%%%%%%%%%%%%%%%%%%%%%%%%%%%%%%%
\begin{document}

\pagenumbering{roman}

%%%%%%%%%%%%%%%%%%%%%%%%%%%%%%%%%%%%%%%%%%%%%%%%%%%%%%%%%%%%%%%%%%%%%%%%%%%%%%%%
%%% title %%%%%%%%%%%%%%%%%%%%%%%
%%%%%%%%%%%%%%%%%%%%%%%%%%%%%%%%%%%%%%%%%%%%%%%%%%%%%%%%%%%%%%%%%%%%%%%%%%%%%%%%

%%%
% title page
\begin{titlepage}
\thispagestyle{empty}
\vspace{-1.5cm}
\includegraphics[height=1.5772cm]{fig/lnm}
\hfill
\includegraphics[height=1.5772cm]{fig/tum_text}

\vfill

\begin{center}
\Huge{\thesistitle}
\\
\vspace{0.2cm}
\LARGE{\Author}
\\
\vspace{0.2cm}
\large{\thesistype}
\end{center}
% \vfill
\vspace{0.1cm}
%
% und an diese Stelle kommt ein tolles Bild aus der eigenen Arbeit
\begin{center}
\includegraphics[width=10cm]{fig/nilifem}
\end{center}
%
% \vspace{0.1cm}
\vfill
%
\begin{minipage}[c]{1.0\textwidth}
\centering
{\large \scshape Supervisor:}\\
{\Supervisor}\\
\texttt{supervisorname@tum.de}\\
\end{minipage}
%
\vspace{1cm}\\
%
\rule{\textwidth}{1pt}
%
\begin{tabular}[t]{l}
Institute for Computational Mechanics
\\
  Prof.\@ Dr.--Ing.\@ W.\ A.\ Wall
\\
Technische Universit\"at M\"unchen
\end{tabular}
\hfill
\begin{tabular}[t]{r}
%   \copyright Lehrstuhl f\"{u}r Numerische Mechanik
\DateOfSubmission
\\
Boltzmannstra\ss e 15
\\
85748 Garching b. M\"unchen (Germany)\\
\end{tabular}
\rule{\textwidth}{1pt}

\end{titlepage}

% ***************** Deckblatt Fakultaet MW *****************

\cleardoublepage
\insertblankpage
\thispagestyle{empty}
\begin{changemargin}{-1.0cm}{-1.0cm}

\begin{flushright}
	\includegraphics[width=20mm]{fig/tum.png}
\end{flushright}
\vspace*{5mm}

{\sffamily
\Huge {\noindent
	Titel der Arbeit (deutsch und englisch,\\
  wenn Volltext englisch, nur englischer Titel\\ 
  notwendig)}

\vspace*{4cm}

\large {\noindent
	Wissenschaftliche Arbeit zur Erlangung des Grades \\[1mm]
	B.Sc./M.Sc. \\[1mm]
	an der TUM School of Engineering and Design.
}
\vspace*{1.5cm}

{%
\normalsize
\begin{onehalfspacing}
\begin{raggedright}
  \begin{tabular}{ll}
  	\textbf{Themenstellender} & Univ.-Prof.\@ Dr.-Ing.\@ Wolfgang\ A.\ Wall \\
  	& Lehrstuhl f\"ur Numerische Mechanik \\[5mm]
  	\textbf{Betreuer} & Betreuer 1  \\
  	& (ggf. Betreuer 2) \\[5mm]
  	\textbf{Eingereicht von} & Martin Mustermann \\
  	& Musterweg 20 (optional)\\
  	& 80999 M\"unchen (optional)\\
  	& + 49 89 123 456 89 (optional)\\[5mm]
	\textbf{Eingereicht am} & \DateOfSubmission in Garching bei M\"unchen \\
\end{tabular}
\end{raggedright}
\end{onehalfspacing}
}
}
\end{changemargin}
% *********************** Thesis Declaration **********************
\begin{changemargin}{-1.0cm}{-1.0cm}
\cleardoublepage
\insertblankpage
\raggedright
\begin{flushright}
	\includegraphics[width=20mm]{fig/tum.png}
\end{flushright}

\vspace*{1.5cm}

{\LARGE \textsf{\textbf{\noindent 
		Erkl\"arung }}}
\thispagestyle{empty}
\vspace*{1.5cm}

\begin{onehalfspacing}
\begin{raggedright}
\textsf{\noindent
Ich versichere hiermit, dass ich die von mir eingereichte Abschlussarbeit selbstst\"andig verfasst
und keine anderen als die angegebenen Quellen und Hilfsmittel benutzt habe.}

\vspace{1.5cm}
\flushleft
\rule[-0,5cm]{\textwidth}{0,5pt}
\flushleft
\textsf{(Ort, Datum, Unterschrift)}
\end{raggedright}
\end{onehalfspacing}

\end{changemargin}

% % Old version of English declaration

%   Hereby I confirm that this is my own work and I referenced all sources and
%   auxiliary means used and acknowledged any help that I have received from
%   others as appropriate.


%%%%%%%%%%%%%%%%%%%%%%%%%%%%%%%%%%%%%%%%%%%%%%%%%%%%%%%%%%%%%%%%%%%%%%%%%%%%%%%%
% Acknowledgement
%%%%%%%%%%%%%%%%%%%%%%%%%%%%%%%%%%%%%%%%%%%%%%%%%%%%%%%%%%%%%%%%%%%%%%%%%%%%%%%%
\cleardoublepage
\insertblankpage
\section*{Acknowledgement}
If desired.

% falls Arbeit in DEUTSCH verfasst werden soll
% \section*{Danksagung}

%%%%%%%%%%%%%%%%%%%%%%%%%%%%%%%%%%%%%%%%%%%%%%%%%%%%%%%%%%%%%%%%%%%%%%%%%%%%%%%%
% Abstract
%%%%%%%%%%%%%%%%%%%%%%%%%%%%%%%%%%%%%%%%%%%%%%%%%%%%%%%%%%%%%%%%%%%%%%%%%%%%%%%%
\cleardoublepage
\insertblankpage
\section*{Abstract}
The abstract should be provided in German and English.

% falls Arbeit in DEUTSCH verfasst werden soll
% \section*{Zusammenfassung}

%%%%%%%%%%%%%%%%%%%%%%%%%%%%%%%%%%%%%%%%%%%%%%%%%%%%%%%%%%%%%%%%%%%%%%%%%%%%%%%%
% Table of content
%%%%%%%%%%%%%%%%%%%%%%%%%%%%%%%%%%%%%%%%%%%%%%%%%%%%%%%%%%%%%%%%%%%%%%%%%%%%%%%%
\cleardoublepage
\insertblankpage
\tableofcontents

\comment{
% add lists as required - ask your supervisor

%%%%%%%%%%%%%%%%%%%%%%%%%%%%%%%%%%%%%%%%%%%%%%%%%%%%%%%%%%%%%%%%%%%%%%%%%%%%%%%%
% List of figures
%%%%%%%%%%%%%%%%%%%%%%%%%%%%%%%%%%%%%%%%%%%%%%%%%%%%%%%%%%%%%%%%%%%%%%%%%%%%%%%%
\cleardoublepage
\insertblankpage
\listoffigures

%%%%%%%%%%%%%%%%%%%%%%%%%%%%%%%%%%%%%%%%%%%%%%%%%%%%%%%%%%%%%%%%%%%%%%%%%%%%%%%%
% List of tables
%%%%%%%%%%%%%%%%%%%%%%%%%%%%%%%%%%%%%%%%%%%%%%%%%%%%%%%%%%%%%%%%%%%%%%%%%%%%%%%%
\cleardoublepage
\insertblankpage
\listoftables

%%%%%%%%%%%%%%%%%%%%%%%%%%%%%%%%%%%%%%%%%%%%%%%%%%%%%%%%%%%%%%%%%%%%%%%%%%%%%%%%
% List of symbols
%%%%%%%%%%%%%%%%%%%%%%%%%%%%%%%%%%%%%%%%%%%%%%%%%%%%%%%%%%%%%%%%%%%%%%%%%%%%%%%%
\cleardoublepage
\insertblankpage
\section*{List of Symbols}
% either by hand or find fancy macro and let me know :o)
or
% \section*{Nomenclature}

% falls Arbeit in DEUTSCH verfasst werden soll
% \section*{Nomenklatur} or
% \section*{Bezeichnungen, Abk\"urzungen, Vereinbarungen}

}

%%%%%%%%%%%%%%%%%%%%%%%%%%%%%%%%%%%%%%%%%%%%%%%%%%%%%%%%%%%%%%%%%%%%%%%%%%%%%%%%
%%% CONTENT %%%%%%%%%%%%%%%%%%%%%%%%%%%%%%%%%%%%%%%%%%%%%%%%%%%%%%%%%%%%%%%%%%%%
%%%%%%%%%%%%%%%%%%%%%%%%%%%%%%%%%%%%%%%%%%%%%%%%%%%%%%%%%%%%%%%%%%%%%%%%%%%%%%%%
\cleardoublepage
\insertblankpage

\pagenumbering{arabic}

\setboolean{@twoside}{true} 
\fancyhead[RO,LE,LO]{\fancyplain{}{}}
\fancyhead[CE]{\fancyplain{}{\leftmark}}
\fancyhead[CO]{\fancyplain{}{\rightmark}}

%%%%%%%%%%%%%%%%%%%%%%%%%%%%%%%%%%%%%%%%%%%%%%%%%%%%%%%%%%%%%%%%%%%%%%%%%%%%%%%%
% Section 1
%%%%%%%%%%%%%%%%%%%%%%%%%%%%%%%%%%%%%%%%%%%%%%%%%%%%%%%%%%%%%%%%%%%%%%%%%%%%%%%%
\section{Introduction}
\label{sec:intro}

The purpose of this document is two-fold:\\
Firstly, this is a plain and simple template with the most basic
\LaTeX-commands and structures introduced that are needed for writing a thesis
in \LaTeX. It should be sufficient for 95\% of the theses submitted at LNM -
this means that completeness is not claimed. It is recommended to stick to
these basic suggestions, knowing that other - possibly better - philosophies
or styles exist. In case a more complex structure is needed, refer to
literature or the web, although you should think twice about introducing a
much more complex structure. \\ 
Secondly, this report gives some basic suggestions on what the structure of
the report should look like and also some brief description about typically
expected content.
\newline
\textbf{Please note}: 
\begin{itemize}
 \item Significant changes to the document style, layout and structure should only be done according to prior agreement with your supervisor.
 \item compile the template with the following command \\
 \verb!latex LaTeX_template_v03.tex && dvipdfm LaTeX_template_v03.dvi!
 \item if warning ``undefined references'' is thrown, include references in \BibTeX-file
   and run command \\
 \verb!bibtex LaTeX_template_v03!
\end{itemize}



%%%%%%%%%%%%%%%%%%%%%%%%%%%%%%%%%%%%%%%%%%%%%%%%%%%%%%%%%%%%%%%%%%%%%%%%%%%%%%%%
% Section 2
%%%%%%%%%%%%%%%%%%%%%%%%%%%%%%%%%%%%%%%%%%%%%%%%%%%%%%%%%%%%%%%%%%%%%%%%%%%%%%%%
\cleardoublepage
% \insertblankpage
\section{Fundamentals of \LaTeX I: structure}
\label{sec:struct}

\subsection{Sections}
\label{sec:sec}
The following section type may be used:

\begin{verbatim}
  \section[TitleInTOC]{FullTitle}\label{sec:SectionLabel}
  \subsection[TitleInTOC]{FullTitle}\label{sec:SubsectionLabel}
  \subsubsection[TitleInTOC]{FullTitle}\label{sec:SubsubsectionLabel}
\end{verbatim}
If case you need even more structure depth, you can also use

\begin{verbatim}
  \paragraph{Title}
  \subparagraph{Title}.
\end{verbatim}
However, you cannot reference (sub-)paragraphs. Think about whether you really
need them or if changing the structure might be better. Making it less complex
usually makes it easier to read (and write!).

It is recommended to write special characters like the German \ss, \"a,
\"o, \"u, etc. in \TeX-code without any packages, i.e. \verb!\ss!, \verb!\"a!,
\verb!\"o!, and \verb!\"u!. This in general prevents difficulties if somebody
else is compiling parts of your work who might not use those packages. You
will quickly get used to it!

\subsection{Floats}
\label{sec:float}


\subsubsection{Figures}
\label{sec:fig}

Figures are to be included in the float environment \verb!\figure! as follows

\begin{verbatim}
  \begin{figure}[placement specifier]
    \centering
    \includegraphics[width=0.3\textwidth]{NameOfGraphic}
    \caption{Description of graphic.}
    \label{fig:GraphicLabel}
  \end{figure}.
\end{verbatim}

The size of the graphic (height or width) should be defined relative to the
textheight or -width, respectively. Two figures next to each other are
included by 

\begin{verbatim}
  [...]
  \includegraphics[width=0.3\textwidth]{NameOfGraphic_1}~
  \includegraphics[width=0.3\textwidth]{NameOfGraphic_2}
  [...],
\end{verbatim}
where as figures on top of each other are included by

\begin{verbatim}
  [...]
  \includegraphics[width=0.3\textwidth]{NameOfGraphic_1}\\
  \includegraphics[width=0.3\textwidth]{NameOfGraphic_2}
  [...].
\end{verbatim}

The placement specifier defines where the figure should be placed
\emph{approximately}. This can be \verb!h!, \verb!t!, \verb!b!, or \verb!p!
which stands for \verb!here!, \verb!top of page!, \verb!bottom of page!, or
\verb!own page for floats!. In 90\% of the cases, \LaTeX~knows best where to
put the floats - leave it out. Trying to put the figures exactly where you
want to put them using \verb!h! is a time waster and counts as
procrastination. The history of students - including me - trying to be smarter
than \LaTeX~is as old as \LaTeX~itself - if not older. Nobody succeeded. 
If you do decide to use it after all, try it right before printing the (very, very) final version.

Note that no endings should be set to the graphic's name. Compiling with
\verb!latex! will automatically choose vector graphics with this name, in
contrast to \verb!pdflatex! that will include raster graphics. Let
\LaTeX~choose the appropriate format - just make sure that you provide a
graphic of an appropriate format in the given graphic path (see "Lead up" in
header file).

\subsubsection{Tables}
\label{sec:tab}

Similar to section \ref{sec:fig}, tables are to be included in a float
environment, namely \verb!table!, as follows:

\begin{verbatim}
  \begin{table}[placement specifier]
    \begin{tabular}{rcl}
      line 1 column 1 & column 2 $ column 3\\
      line 2 column 1 & column 2 $ column 3
    \end{tabular}
    \caption{Description of table.}
    \label{tab:TableName}
  \end{table}
\end{verbatim}

\verb!rcl! stands for \verb!right!, \verb!center!, and \verb!left! and
referres to the alignment of the column. "\verb!|!" inbetween will separate
the columns by a vertical line. \verb!\hline! after, e.g. \verb!\\! inserts a
horizontal line after the line (see also table \ref{tab:labelspec}
below). \LaTeX-tables are far from perfect if your table gets a bit more
complex, e.g. when inserting bigger equations. Formatting tables to someones
total satisfaction can be very tricky and trying to be too fancy is again
mostly a time waster. Keep it simple!

\subsection{Equations}
\label{sec:eq}

Equations with numbering should be included by the \verb!align! environment
like this
\begin{verbatim}
  \begin{align}\label{eq:EquationName}
    f(x) := \exp(x) &= \sum_{n=0}^{\infty} \frac{x^{n}}{n!} \nonumber \\
                    &\approx 1 + x + \frac{x^{2}}{2} + \frac{x^{3}}{6}
  \end{align}
\end{verbatim}
for
~
\begin{align}\label{eq:EquationName}
  f(x) := \exp(x) &= \sum_{n=0}^{\infty} \frac{x^{n}}{n!} \nonumber \\
                  &\approx 1 + x + \frac{x^{2}}{2} + \frac{x^{3}}{6}.
\end{align}

Use \verb!\\! to insert a multiline equation. Use \verb!&! to align those 
to each other as shown. It defines one \verb!rl!-aligned pair, which means that
everything that stands on the left of \verb!&! will be right align and vice
versa.  The equation itself will be centered. Use \verb!\nonumber! to surpress
the numbering of certain lines as shown. If no numbering is wanted use
\verb!align*! and leave out the label. Look on the web for more complex
alignments - see also the \verb!alignat!-environment that could be used
alternatively. Use it scarcely. \verb!align! is powerfull and easy to use; it
should be used most (meaning 99\%) of the times.

Objects like vectors, matrices, and similar things should be formated
consistently using a dedicated command defined in the beginning of youe
\LaTeX-file. See for instance above the definition of \verb!\vec! for writing
a vector $\vec{x}$. Thus, you can also read your equations in \LaTeX-code more
easily. Additionally the nomenclature of an object is changed by
changing only one line in your header. 

It is recommended to insert sub- and superscript as well as any manipulation of
symbold like \verb!\hat! etc. by using \verb!{}! even when only one symbol is
manipulated, e.g. \verb!\hat{\vec{x}}_{i}! for $\hat{\vec{x}}_{i}$. It's good
coding and can otherwise also be an ever recurring source of typos and
compilation errors.


%%%%%%%%%%%%%%%%%%%%%%%%%%%%%%%%%%%%%%%%%%%%%%%%%%%%%%%%%%%%%%%%%%%%%%%%%%%%%%%%
% Section 3
%%%%%%%%%%%%%%%%%%%%%%%%%%%%%%%%%%%%%%%%%%%%%%%%%%%%%%%%%%%%%%%%%%%%%%%%%%%%%%%%
\cleardoublepage
\section{Fundamentals of \LaTeX II: referencing}\label{sec:referencing}

\subsection{Sections, equations, figures, and tables}
\label{sec:refseceq}

Everything that can be referenced should have a label. Otherwise use the
non-referenceable version, e.g. the \verb!align*! environment instead of
\verb!align!. This also means every proper section of your thesis should have a
label. Furthermore, every equation that has a number should be important
enough to be referenced and thus gets a label. Sometimes this can also be a
sentence like ``Equation (4) states the key result of this section.'' to
emphasise its importance. If the equation is not referenced, it is not
important and should not even have a number - use \verb!align*!. Figures and
tables always have a label und should always be referenced at least
once. Some readers don't look at them before they are referenced in the
text. Of course, things that cannot be referenced should
never have a label. 

The label should have a certain structure and start with a specifier
according to the object it references, followed by a colon and the actual name
of the label. An example would be \verb!\label{sec:intro}! for the label of
the first section ``Introduction''. The specifiers are given in table
\ref{tab:labelspec}. Referencing a labelled objects is done by the command
\verb!\ref!, e.g. \verb!\ref{sec:intro}!. Equations should always be
referenced by \verb!\eqref!, which automatically puts the brackets around the
equation number (comes with an ams-package already included in this
template).

\begin{table}[h]
  \centering
  \begin{tabular}{|l|l|}\hline
    \bf object & \bf specifier \\\hline
    (sub-)section & sec\\
    equation & eq\\
    figure & fig\\
    table & tab\\ 
    (appendix) & (app)\\ \hline
  \end{tabular}
  \caption{Object specifiers for labels.}
  \label{tab:labelspec}
\end{table}

\subsection{Citations}
\label{sec:cite}

For citations you have to use \BibTeX. For citing a reference, you use
\verb!\cite{LabelOfRef}! As label, it is suggested to use the surname of the
first author, directly followed by the year, and, if there is more than one
publication of that author in the that year, a lowercase letter in
alphabetical order. For instance: \verb!\cite{Brooks1982}! for article
\cite{Brooks1982}. Others might prefer other systems, which is fine as long
as it is consistent. If you use only a few references, adding them by hand to
your \BibTeX-file seems the easiest. If there are more references you
might want to create your \BibTeX-file by a reference software like Jabref.



%%%%%%%%%%%%%%%%%%%%%%%%%%%%%%%%%%%%%%%%%%%%%%%%%%%%%%%%%%%%%%%%%%%%%%%%%%%%%%%%
% Section 4
%%%%%%%%%%%%%%%%%%%%%%%%%%%%%%%%%%%%%%%%%%%%%%%%%%%%%%%%%%%%%%%%%%%%%%%%%%%%%%%%
\cleardoublepage
\section{The actual structure of your report} %Your new development/method/finding
\label{sec:developments}

This section is added to show what the complete structure of your thesis
might look like. Coincidentally, this is also the typical structure of a
scientific report, and hence important to know anyway. The typical sections
are discussed below, but some specifics might vary according to the type of
research project. Please discuss the actual structure of your thesis with your
supervisor. 

\subsection{Abstract}
\label{sec:Abstract}
This is the most often read section. It will determine if someone actually
bothers to read more of what you've written. Only write this section after you
have completed your report. The abstract consist of only one paragraph, mostly
limited to 200-300 words. Literally, a summary of your work. Write a sentence
or two about each of the main sections of your report, as discussed in this
section. When summarizing results, make the reader aware of the most important
results (including numbers when applicable) and important conclusions or
questions that follow from these.


\subsection{Introduction}
\label{sec:Introduction}
The objective of writing this section is to introduce the reader to
the problem and also show that you understand the problem statement
in the context of current knowledge in the field.

Any problem in science can be introduced by simply including the following
sections:
\begin{enumerate}
\item Give background about the problem you are investigating. 
\item Describe what has been done up to now to address this problem (and
also refer to the literature in which this work has been done). 
\item State the objective for taking on the current study or the hypothesis
that will be tested in the current study. 
\item Provide a single paragraph of the contents of the rest of the article
(focusing on that which is reported in \textquotedblleft{}Materials and Methods\textquotedblright{},
\textquotedblleft{}Results\textquotedblright{}, \textquotedblleft{}Discussion\textquotedblright{}
and \textquotedblleft{}Conclusions\textquotedblright{}).
\end{enumerate}

\subsection{Materials and Methods}
\label{sec:MaterialsAndMethods}

Describe the relevant materials, methods, tools, equipment, software,
hardware, experimental setup, experimental conditions, general procedures,
etc. so that someone else can repeat your study or judge the scientific
merit thereof. Do not describe everything you did, or who did what
(unless specified or relevant). This section is not a set of instructions! It rather describes the complete methodology used, so that it can be reproduced by someone of similar skill.
Keep it as concise as possible. Also provide the reader the name of
the company that produced relevant equipment/software
used and the country where the company is located.

\subsection{Results}
\label{sec:Results}

Here you must simply report on the results that you achieved. When
reporting the results, provide a context to the reader, such as to
describe the question that was addressed by reporting a specific result.
Do not interpret any results here \textendash{} leave this for the
discussion! Do not show the same results in a figure and a table \textendash{}
choose the one which can best communicate your results. The most important
aspect(s) of the results reported in a table or figure should be reported
in the text while also referring to the table or figure. In other
words, the text should complement the tables and figures. Results
reported here can be raw data (obtained from instruments), converted
data (obtained after converting raw data as described in the \textquotedblleft{}Materials
and Methods\textquotedblright{} section) or applicable numerical examples.


\subsection{Discussion}
\label{sec:Discussion}

Here you must provide an interpretation of your results and support
for all your conclusions using what you have at your disposal \textendash{}
your results and generally accepted knowledge, such as published literature
(if needed). Also describe the significance of your findings clearly.
Commenting on the methods that you employed might also be relevant. 

If your results agree with what you expected, describe the theory or
previously described observations that your results delivered. If your results
do not agree with what expected, explain why this might have happened. A lot
can be learned from what did not work! You may also deduce alternative
explanations if reasonable ones exist. Understanding and interpreting what the
limitations of your study was or what went wrong is crucial for the general
increase of scientific knowledge -- and of course your own knowledge. Do not
neglect this aspect of your discussion. Do not just dismiss your results as
useless or inconclusive if it does not clearly align with your initially
stated objective/hypothesis. Since you were doing very good scientific
research, appropriately report on the limitations of your study, so that
these limitations do not come across as shortcomings, thereby deeming your
work of lower quality. Therefore, attempt to end your discussion section by
motivating the significance of your results, even if your results have been
shown to not be statistically significant/aligned with the initial
objectives. 


\subsection{Conclusions}
\label{sec:Conclusions}

Concluding remarks should naturally come from the discussion described
above. This is the \emph{Grand Finale} of your report. This section consists
of 2 parts at most. In the first part, repeat your most important finding. The
last part should suggest existing challenges, potential future directions or
recommendations for further studies. Keep it concise!


\subsection{References}
\label{sec:References}

In all of the sections prior to this section, a citation is made in the text
which refers to a reference -- i.e. a source where this information comes
from. The full details of these references are then listed in the this section
-- the references. 


Different styles of citation and accompanying referencing exist. The style
refers to the order in which the author names are listed, information of the
cited reference to be listed both in the text and also in the references
section, formatting, punctuation etc., which should be consistently used
throughout the report. The style to be used depends on the type of
publication, e.g. book, journal, report, newspaper article, website.

For your report to be submitted to LNM, you must use the numerical system when
you use any references in your work. This system of referencing basically
consist of placing a number in-text (starting at ``1'', of course) next to
where you want to indicate a reference.  This reference is then fully
described in the ``References'' section. 

%%%%%%%%%%%%%%%%%%%%%%%%%%%%%%%%%%%%%%%%%%%%%%%%%%%%%%%%%%%%%%%%%%%%%%%%%%%%%%%%
%%% Beginn des Anhangs %%%%%%%%%%%%%%%%%%%%%%%
%%%%%%%%%%%%%%%%%%%%%%%%%%%%%%%%%%%%%%%%%%%%%%%%%%%%%%%%%%%%%%%%%%%%%%%%%%%%%%%%
\cleardoublepage
\appendix{}
\section{First Appendix}
% \newpage
\section{Second Appendix}



%%%%%%%%%%%%%%%%%%%%%%%%%%%%%%%%%%%%%%%%%%%%%%%%%%%%%%%%%%%%%%%%%%%%%%%%%%%%%%%%
%%% Literaturverzeichnis %%%%%%%%%%%%%%%%%%%%%%%
%%%%%%%%%%%%%%%%%%%%%%%%%%%%%%%%%%%%%%%%%%%%%%%%%%%%%%%%%%%%%%%%%%%%%%%%%%%%%%%%
\cleardoublepage

%%
%% BibTeX users:
\bibliographystyle{plain}
\bibliography{template.bib}

%%%%%%%%%%%%%%%%%%%%%%%%%%%%%%%%%%%%%%%%%%%%%%%%%%%%%%%%%%%%%%%%%%%%%%%%%%%%%%%%
%%%%%%%%%%%%%%%%%%%%%%%%%%%%%%%%%%%%%%%%%%%%%%%%%%%%%%%%%%%%%%%%%%%%%%%%%%%%%%%%
%%% END DOCUMENT
%%%%%%%%%%%%%%%%%%%%%%%%%%%%%%%%%%%%%%%%%%%%%%%%%%%%%%%%%%%%%%%%%%%%%%%%%%%%%%%%
%%%%%%%%%%%%%%%%%%%%%%%%%%%%%%%%%%%%%%%%%%%%%%%%%%%%%%%%%%%%%%%%%%%%%%%%%%%%%%%%
\end{document}
